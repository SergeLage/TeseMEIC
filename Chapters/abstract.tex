\abstractEN % Do NOT modify this line

Portugal is a country historically linked to the sea, fishing being a very important activity for the Portuguese economy. On the other hand, tax fraud is a very present on fishery an other economic activity's and harmful phenomenon for Portugal. For that reason there is a need to create ways to inspect this activity more efficiently. With the constant evolution of new technologies, this need is increasingly close to being met in a concrete way.
In order to contribute to the resolution of this problem the objectives of this dissertation are to demonstrate the possibility of classifying when the vessels are fishing and when they fish in an area other than the usual one, this using only speed and location data as the first objective. The second objective is to classify the fishing activity with the same data, speed and location, and to compare with the assigned license.
There are several studies in this area all over the globe. What makes this work unique is the use of data created by a device on board in which there is no human interference. %the desired effect to the system we intend to create.%
The data used by this solution is produced by the system MONICAP, a bluebox system, mandatory for all vessels over 12 meters in the European Union. This system records speed, heading and location data.
For the first problem, create a machine learning system to identify whether
the vessel is fishing using speed and using clustering algorithms to identify whether
the fishing zone is usual or not. This system is designed so that it can be integrated into MONICAP itself. For the second problem, we will use data mining
methods to correlate the data provided by MONICAP and the type of fishing
license.
The models were tested and evaluated using well-established data mining techniques.
The use of velocity turns out to be enough to create systems capable of satisfying
the proposed objectives, so goals are all achieved.


%The main problem pointed in this work is the lack of VMS data by an authority.



%MONICAP is a monitoring system for the inspection of fishing using the Global Positioning System (GPS) for vessel location and Inmarsat-C technology for satellite communications between ships and a ground control center. MONICAP was successfully introduced on the market by Xsealence and is currently installed or currently being installed on about 800 fishing vessels operating under the control of the authorities of Portugal, Spain, France, Ireland, and Angola. Within the scope of this Master's thesis, it is proposed to use Portuguese fishing data from the Vessel Monitoring Systems to extract patterns of behavior related to the fishing zones, times, speeds, and directions of the course performed by the ships. The descriptive statistical analysis of these makes it possible to identify patterns of fishing activity, as well as the identification of outliers. The identification of outliers, when performed in real-time, will lead to the consequent generation of alerts. The present study represents the first comprehensive approach to detect and identify the behavior of fishing activities for the main types of equipment based on the tracking of Portuguese fishing fleet data.
%
%The objectives of this work are to identify a solution capable of classifying through the data generated by Monicap if the vessel is fishing and if so, in a new area. Also, create a model capable of classifying the fishing license with the data generated in MONICAP.
%
%For the first problem, we will use a machine learning system to identify whether the vessel is fishing using speed and using clustering algorithms to identify whether the fishing zone is usual or not.
%This system will be designed so that it can be integrated into MONICAP itself.
%For the second problem, we will use data mining methods to correlate the data provided by MONICAP and the type of fishing license.
%
%The use of velocity turns out to be enough to create systems capable of satisfying the proposed objectives.
%The main problem pointed in this work is the lack of VMS data by an authority.



% Keywords of abstract in English
\begin{keywords}
Vessel Monitoring System, Data Mining, Fishing
\end{keywords} 
