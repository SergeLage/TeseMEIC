\chapter{Arquivo.pt Infrastructure}

This chapter describe an overall description about Arquivo.pt infrastructure, its data and how it is obtained, in order to provide a systems enviroment context, and data provenance context.

\section{System Components Description}


\begin{figure}[H]
    \centering
    \includegraphics[width=1\linewidth]{Chapters/img/arquivo_infrastrcture_v2.pdf}
    \caption{Arquivo.pt Infrastructure.}
    \label{fig:arquivo_infrastructure}
\end{figure}

Arquivo.pt infrastructure is composed by 4 key system components, each one with their own and unique functionality. These systems working toghether provide the Arquivo.pt overall functionality.

\subsection{Crawler System}

The crawler system job is to harvest the web collecting and storing its contents. It uses crawlers such as Heritrix to perform this task. The crawlers are configured with a list of starting URLs (seeds) and with a specific scope and budget. An example of configuration is for instance to crawl only URLs from \emph{.pt} top-level domain and to download up to 10~000 URLs per host, for performance reasons. It then starts to crawl those seeds, discovering more URLs to crawl and preserve them.
Other crawler used by the system is the Brozzler~\cite{brozzler}, that uses Chromium instances to render the web pages instead of only fetching the resource's representation. These crawlers can discover and preserve more content with higher quality with the downside that they demand more hardware resources, the process is also slower. All contents that are fetched by those crawlers are stored in ARC and WARC file formats to be used by other systems.

\subsection{Replay System}

The replay system is responsible to reproduce the preserved contents page back to the user trying to provide a similar experience and navigability as the original web page. The software used for this kind of task is commonly named Wayback Machine. There are several implementations of Wayback Machines, for instance OpenWayback~\cite{openwayback} written in Java. Arquivo.pt uses for this purpose the PyWB~\cite{pywb} a Wayback Machine written in Python. These systems use special indexes named CDX~\cite{cdxpage} that map the URLs and their position within the ARC/WARC files. % TODO more about this

\subsection{Indexing System}

The Indexing System is responsible for all the index work so that the contents preserved can be searched and reproduced. It is composed by a Hadoop\footnote{\url{https://hadoop.apache.org/}} Cluster that processes the Terabytes of ARC/WARC files stored by the Crawler system and produce Lucene\footnote{\url{https://lucene.apache.org/}} indexes to be used by the full-text Search System and CDX indexes to be used by the Replay System.

\subsection{Text Search System}

The Text Search System is responsible to provide a way to find information at Arquivo.pt. Traditionally, Web Archives only allow to search for a web page through the URL. Arquivo.pt provides more search capabilities allowing users to search through query terms. The system will display preserved web pages such that their contents are related with the submitted terms. This system uses a modified NutchWAX\cite{nutchwax} implementation and the Lucene indexes generated by the Indexing System.


\subsection{Image Search System}

The Image Search System is the new feature that is being developed and it provides the components to enable image search at Arquivo.pt. Its componsed by an Image Search WebApp that answer the the image search requests that come from Arquivo.pt frontend and uses Solr to answer the query. The indexes to the system are created at the Indexing System similary from the Text Search System but with a different hadoop job. Is at this job that the classification system will need to integrated.


\section{Arquivo.pt Data Characterization}
% TOTAL AMOUNTS of Arquivo.pt
% (Atualizar valor na altura) 
As of 8 June 2018, Arquivo.pt has a total amount of 4~533~859~612 preserved web files, gathered in 2~160~318 Archive file format (ARC) and Web ARChive file format (WARC)~\cite{isowarc} compressed files fulfilling 222.4 Terabytes of disk space storage.
% referenciar ISO, por rodape Web ARChive format?

The ARC/WARC file formats specify a method for combining multiple digital resources into an aggregate archival file together with metadata information. Each WARC file is a concatenation of one or more WARC records. Each WARC record consisting of a header with metadata information followed by a content block with the corresponding resource, such as images, text documents or any type of resource found on the Web.

The information preserved by Arquivo.pt is gathered automatically using web crawlers. This web crawlers ran within a specific scope, and all the information they collect is kept for preservation.

There are four different web crawlers scopes:
\begin{itemize}
    \item Broad Domain - top-level domain \emph{.pt} crawls and suggested websites.
    \item Daily - daily crawls of Portuguese news websites.
    \item Special - thematic crawls, for instance Portuguese elections and top-level domain \emph{.eu} crawl.
    \item High Quality  - high quality crawls to selected websites.
\end{itemize}

Each crawl has different configurations. For instance, the broad domain crawlers don't download files from the web with more than 10 MB, but the special crawls and high-quality crawls don't have those restrictions. Moreover, they are configured to accept all mime-types found on the Web. For this reason, the type of data that can be found at Arquivo.pt can be very widespread and heterogeneous.

% example of WARC record
Table~\ref{tbl:general-mime-types} reports the top 10 mime-types found during a 2017 \emph{.pt} top-level domain crawl. The presented mime-types are reported by the web servers using the HTTP meta information when each resource is collected. It is important to note that sometimes the reported mime-type is wrong regarding the real resource type that the web server is providing, although these situations are not very common.

% EXTENDER para todas as coleções do Arquivo.pt
% TOP 10 MIME TYPES TOTAL:298887561
\begin{table}[H]
\centering
\caption[Resources distribution by mime-type.]{Measure of the number of resources mime-types collected on Arquivo.pt last broad domain crawl.} % AWP24
\label{tbl:general-mime-types}
\begin{tabular}{|c|c|c|}
\hline
\textbf{\% Amount}  & \textbf{Number of URLs}   & \textbf{mime-types}               \\ \hline
79.60               &   237966251               & text/html                         \\ \hline
10.03               &   29997689                & image/jpeg                        \\ \hline
2.45                &   7328179                 & image/png                         \\ \hline
0.77                &   2305834                 & application/pdf                   \\ \hline
0.76                &   2271770                 & application/javascript            \\ \hline
0.72                &   2145481                 & text/xml                          \\ \hline
0.62                &   1869902                 & application/rss+xml               \\ \hline
0.62                &   1861165                 & application/json                  \\ \hline
0.60                &   1820236                 & image/gif                         \\ \hline
0.58                &   1783630                 & text/css                          \\ \hline
3.25                &   1783630                 & all others                        \\ \hline
\end{tabular}
\end{table}

Table~\ref{tbl:image-types} presents the top 8 mime-types regarding image contents that were found during the same crawl. The most common image type is the \emph{jpeg} with 75\% of the total images followed by \emph{png} with 18\%. Although only 4\% of the images are \emph{gif}, this type can present an extra challenge to classify because the images have an animation.

%Comparing to a more recent crawl from 2017, where 79\% of the collected resources are text/html and 10.03\% are image/jpeg, followed by 2.45\%.
% isto não deve estar bem
\begin{table}[H]
\centering
\caption{Mime-types distribution for image type contents.}
\label{tbl:image-types}
\begin{tabular}{|c|c|c|}
\hline
\textbf{\% Amount} & \textbf{Number of URLs} & \textbf{mime-types} \\ \hline
75.22              & 29997689                & image/jpeg          \\ \hline
18.37              & 7328179                 & image/png           \\ \hline
4.56               & 1820236                 & image/gif           \\ \hline
0.98               & 389839                  & image/svg+xml       \\ \hline
0.37               & 147849                  & image/jpg           \\ \hline
0.34               & 135720                  & image/x-icon        \\ \hline
0.09               & 38577                   & image/pjpeg         \\ \hline
0.06               & 22717                   & image/bmp           \\ \hline
\end{tabular}
\end{table}

% TOP 10 IMAGES MIME TYPES
% data characterization
Since the origin of the images are from the web and from multiple different websites, these images can have many sizes, different resolutions and any type that is allowed on the web.

