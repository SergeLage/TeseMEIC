\abstractPT  % Do NOT modify this line

O MONICAP é um sistema de monitorização para a inspeção da atividade pesqueira que utiliza o Sistema de Posicionamento Global (GPS) para a localização da embarcação e a tecnologia Inmarsat-C para as comunicações via satélite entre navios e um centro de controle terrestre. O MONICAP foi introduzido com sucesso no mercado pela empresa Xsealence e está atualmente instalado, ou em fase de instalação, em cerca de 800 navios de pesca que operam sob o controlo das autoridades de Portugal, Espanha, França, Irlanda e Angola. No âmbito desta tese de mestrado propõe-se a utilização dos dados de pesca portugueses provenientes dos Sistemas de Monitorização de Embarcações com o objetivo de extrair padrões de comportamento relacionados com as zonas de pesca, os tempos, as velocidades e as direções do percurso efetuado pelos navios. A análise estatística destes viabiliza a identificação de padrões de atividade de pesca, bem como a identificação de outliers. A identificação de outliers quando realizada em tempo real, levará à consequente geração de alertas que podem ser utilizados pelas autoridades competentes potenciando assim o desenvolvimento de meios de controlo de atividades ilícitas. O presente estudo representa a primeira abordagem abrangente com o objetivo de detetar e identificar o potencial comportamento das atividades de pesca, para os principais tipos de equipamento, baseado no rastreamento de dados de frota pesqueira portuguesa.


% Keywords of abstract in Portuguese
\begin{keywords}
Vessel Monitoring System, Mineralização de dados, Pescas
\end{keywords}
% to add an extra black line
