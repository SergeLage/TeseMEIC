\abstractPT  % Do NOT modify this line

Portugal é um país historicamente ligado ao mar, sendo a pesca uma atividade muito importante para a economia portuguesa. Por outro lado, a fraude fiscal está muito presente na pesca, e outras atividades económicas, é um fenómeno prejudicial para Portugal. É necessário inspecionar essa atividade com maior eficiência. Com a constante evolução das novas tecnologias, essa necessidade está cada vez mais próxima de ser atendida de maneira concreta.
Para contribuir com a solução deste problema, os objetivos desta dissertação são classificar quando os navios estão a pescar e quando pescam numa área não habitual, apenas com uso de dados de velocidade e localização como primeiro objetivo. O segundo objetivo é classificar a atividade de pesca com os mesmos dados, velocidade e localização, e comparar com a licença atribuída.
Existem vários estudos nessa área em todo o mundo. O que torna esse trabalho único é o uso de dados criados por um dispositivo a bordo sem interferência humana. 
Os dados utilizados por esta solução são produzidos pelo sistema MONICAP, um sistema bluebox, obrigatório para todos os navios com mais de 12 metros na União Europeia. Este sistema regista dados de velocidade, rumo e localização.
Para o primeiro problema, criar um sistema de aprendizagem automatica para identificar se o navio está pescando usando velocidade e com algoritmos de agrupamento identificar se a zona de pesca é a habitual. Este sistema foi projetado para ser integrado no próprio MONICAP. Para o segundo problema, usamos métodos de mineração de dados para correlacionar os dados fornecidos pelo MONICAP e o tipo de pesca licença.
Os modelos foram testados e avaliados usando técnicas de mineração de dados bem estabelecidas.
O uso da velocidade acaba sendo suficiente para criar sistemas capazes de satisfazer
objetivos propostos, para que todos os objetivos sejam alcançados.


%O MONICAP é um sistema de monitorização para a inspeção da atividade pesqueira que utiliza o Sistema de Posicionamento Global (GPS) para a localização da embarcação e a tecnologia Inmarsat-C para as comunicações via satélite entre navios e um centro de controle terrestre. O MONICAP foi introduzido com sucesso no mercado pela empresa Xsealence e está atualmente instalado, ou em fase de instalação, em cerca de 800 navios de pesca que operam sob o controlo das autoridades de Portugal, Espanha, França, Irlanda e Angola. No âmbito desta tese de mestrado propõe-se a utilização dos dados de pesca portugueses provenientes dos Sistemas de Monitorização de Embarcações com o objetivo de extrair padrões de comportamento relacionados com as zonas de pesca, os tempos, as velocidades e as direções do percurso efetuado pelos navios. A análise estatística destes viabiliza a identificação de padrões de atividade de pesca, bem como a identificação de outliers. A identificação de outliers quando realizada em tempo real, levará à consequente geração de alertas que podem ser utilizados pelas autoridades competentes potenciando assim o desenvolvimento de meios de controlo de atividades ilícitas. O presente estudo representa a primeira abordagem abrangente com o objetivo de detetar e identificar o potencial comportamento das atividades de pesca, para os principais tipos de equipamento, baseado no rastreamento de dados de frota pesqueira portuguesa.Os objetivos deste trabalho é identificar uma soluçao capaz de classificar atraves dos dados gerados pela Monicap se a embarcaçao se encontra a pescar e se sim, numa area nova. Tambem, criar um modelo capaz de classificar a licensa de pesca com os dados gerados no MONICAP.Para o primeiro problema vamos utilizar um sistema de aprendisagem automatica para identificar se a embarcação esta em atividade utilizado a velociade e atraves de algoritmos de clustering identificar se a zona de pesca é habitual ou não.Este sistema sera desenhado de forma a poder ser integrado no proprio MONICAP.Para o segundo problema iremos utilizar metodos de data mining para coorelacionar os dados fornecidos pelo MONICAP e o tipo de liçensa de pesca.%

% Keywords of abstract in Portuguese
\begin{keywords}
Vessel Monitoring System, Mineralização de dados, Pescas
\end{keywords}
% to add an extra black line
