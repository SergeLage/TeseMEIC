\abstractPT  % Do NOT modify this line

Portugal é um país historicamente ligado ao mar, sendo a pesca uma atividade muito importante para a economia portuguesa. Por outro lado, a fraude fiscal está muito presente na pesca, e outras atividades económicas, é um fenómeno prejudicial para Portugal. É necessário inspecionar essa atividade com maior eficiência. Com a constante evolução das novas tecnologias, essa necessidade está cada vez mais próxima de ser atendida de maneira concreta.
Para contribuir com a solução deste problema, os objetivos desta dissertação são classificar quando os navios estão a pescar e quando pescam numa área não habitual, apenas com uso de dados de velocidade e localização como primeiro objetivo. O segundo objetivo é classificar a atividade de pesca com os mesmos dados, velocidade e localização, e comparar com a licença atribuída.
Existem vários estudos nessa área em todo o mundo. O que torna esse trabalho único é o uso de dados criados por um dispositivo a bordo sem interferência humana. 
Os dados utilizados por esta solução são produzidos pelo sistema MONICAP, um sistema bluebox, obrigatório para todos os navios com mais de 12 metros na União Europeia. Este sistema regista dados de velocidade, rumo e localização.
Para o primeiro problema, criar um sistema de aprendizagem automatica para identificar se o navio está pescando usando velocidade e com algoritmos de agrupamento identificar se a zona de pesca é a habitual. Este sistema foi projetado para ser integrado no próprio MONICAP. Para o segundo problema, usamos métodos de mineração de dados para correlacionar os dados fornecidos pelo MONICAP e o tipo de pesca licença.
Os modelos foram testados e avaliados usando técnicas de mineração de dados bem estabelecidas.
O uso da velocidade acaba sendo suficiente para criar sistemas capazes de satisfazer
objetivos propostos, para que todos os objetivos sejam alcançados.

% Keywords of abstract in Portuguese
\begin{keywords}
Vessel Monitoring System, Mineralização de dados, Pescas
\end{keywords}
% to add an extra black line
