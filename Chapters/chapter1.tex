

% 
% chapter1.tex
% ThesisISEL
% 
% Created by Serge Lage on 2019/07/30.
%
% ================
% = Introduction =
% ================
\chapter{Introduction}
\label{cha:introduction}
This chapter introduces the motivation, context and the goals of this work. Finally, it presents the overall structure of this document.

\section{Motivation} % (fold)
\label{sec:motivation}

\subsection{Fishing Activity} % (fold)
\label{sub:fishing_activity}
The increased fishing activities mankind imposed on the marine ecosystems is a threat for the future sea economy and for the marine ecosystem's integrity \cite{AgardyEffects}.\\
Fisheries mapping is needed for implementing better ecosystem management and to secure a healthy marine population \cite{AlfredImperative}.
The fishing activity represents an important activity for the Portuguese economy. In the European Union, Portugal is the country that has the highest consumption of fish per person and the third worldwide needing 55,6 kg (per capita/year), \cite{WEBSITE:ConsumoPescasPortugal}. Portugal is a country connected to the sea by its great coast, all along with its continental territory, almost all the coastal villages have a fishing community.
An important issue of concern to the authorities is the occurrence of tax evasion that continues to cause damage to the Portuguese economy. In Portugal, the estimated tax evasion in all economic activities represents 21,9\% of it's Gross Domestic Product (GDP) \cite{BOOK:EsbocoFraude}.
The use of statistical pattern recognition techniques to analyze the data makes possible to identify who operates in the margin of the law more rapidly and methodically \cite{ShuklaBigData}. Reducing tax evasion will allow strong gains and potentially will develop the economy, making the fishing activity more just for everyone involved in this activity.

MONICAP \cite{WEBSITE:MonicapXsealence} is a monitoring system for the inspection of fishing using the Global Positioning System (GPS) for vessel location and Inmarsat-C \cite{WEBSITE:inmarsatC} technology for satellite communications between ships and a ground control center. MONICAP was successfully introduced on the market by Xsealence \cite{WEBSITE:Xsealence}) and is currently installed or currently being installed on about 800 fishing vessels operating under the control of the authorities of Portugal, Spain, France, Ireland, and Angola. Within the scope of this Master's thesis, it is proposed to use Portuguese fishing data from the Vessel Monitoring Systems (VMS) to extract patterns of behavior related to the fishing zones, times, speeds, and directions of the course performed by the ships. The descriptive statistical analysis of these makes it possible to identify patterns of fishing activity that can be used for different proposes like sustainable fishing, models for fuel efficiency, and  models to detect illegal activities.

VMS provides a unique and independent method to derive patterns of spatially and temporally explicit fisheries activity. Such information may feed into ecosystem management plans seeking to achieve sustainable fisheries while minimizing potential risk to non-target species (e.g. cetaceans, seabirds, and elasmobranchs) and habitats of conservation concern. With multilateral collaboration, VMS technologies may offer an essential solution to quantifying and managing ecosystem disturbance, particularly on the high-seas.


% section fishing_activity (end)


\subsection{Analytics} % (fold)
\label{sub:analytucs}
The concept of Analytics refers to the ability to use data, perform predictive analytics, and systematic reasoning to improve performance in key business domains and lead to a more efficient decision-making process.
There is extensive use of mathematics and statistics, like descriptive techniques and predictive models, which allow gaining valuable knowledge from the data.
The insights from data are used to recommend action or to guide decision making rooted in a business context.


% section analytics (end)

\subsection{Data mining} % (fold)
\label{sub:data_mining}
Data mining is the process of discovering actionable information from large sets of data. Data mining uses mathematical analysis to derive patterns and trends that exist in data. Typically, these patterns cannot be discovered by traditional data exploration because the relationships are too complex or because there is too much data.


% section data_mining (end)


% section introduction (end)

\section{Goals} % (fold)
\label{sec:objectives}

\textbf{Objective 1: Local tool} \\
In real-time, will be developed an application to be installed in the MONICAP, which allows better describing the fishing zones. At each one of the vessels, this application will work in real-time with the data from the fishing activity of this vessel. This tool will be local and will be used unsupervised techniques of machine learning.
The derived application should be able to identify patterns in two strands:
Speed: identify if the vessel is in fishing activity or not;
Location: identify the usual fishing spots. These spots knowledge to cross-check in real-time if the current location of fishing is new to the vessel.

\textbf{Objective 2: Centralized tool} \\
Using VMS data and information on fishing licenses per vessel, our goal is to design models that are capable of classifying vessels by type of fishing only using VMS data. These models could be used to classify vessels by fishing activity, allowing crossing this classification with the information corresponding to the vessel license.

% section objectives (end)

\section{Document Structure} % (fold)
\label{sec:work_structure}
This document is divided into six main chapters. Chapter 2 gives an overview of the State of the Art concerning the usage of technologies in the fishing industry to detect outliers' behaviors such as activity. In Chapter 3, we focus on the data description and analysis, as well as the needed data pre-processing treatment to find possible solutions for the goals described previously. The data are divided into two categories:
\begin{itemize}
\item VMS Records: data generated by MONICAP system;
\item VMS Vessels: data coming from the vessel's captains, manually inserted.
\end{itemize}
In Chapter 4 were presented two methodology's to classify the data in real-time, only using the available data by the Blue Box:
\begin{itemize}
\item Using velocity data, to classify if the vessel is fishing;
\item Using location data, to classify if the vessel is in activity in a new area;
\end{itemize}
In chapter 5 is presented an approach to answer the second objective: using the data from all the vessels, how to identify fishing activities that are not under the vessel's fishing license. Different data mining methods will be used to derive predictive models. Corresponding results are compared through correct classification performance measures.
Chapter 6 is presented with the results of the validation of the models and methods presented to answer the proposed objectives.
Chapter 7 contains the conclusions obtained during the elaboration of this work.

% section work_structure (end)

\section{Publications} % (fold)
\label{sec:publications}

\subsection{U. C. report to the final project of course (FPC)} % (fold)
\label{sub:fpc}
My FPC entitled "Análise de Padrões para Encontrar Fraude nas Pescas" was developed in the same data analysis context. In that work I tried to solve an analogous problem with data coming from the VMS file, but with a different approach.\\
FPC work was focused on abnormalities regarding the declaration of fish caught, by quantities and type of fish. It was used the data provided by the Capitan with quantities caught per type of fish and used VMS Records data to consider standards, as the time of the year and fishing positions.

% section fpc (end)

\subsection{Published Paper} % (fold)
\label{sub:published_paper}
Fishing Monitor System Data: A Naïve Bayes Approach\\
Authors: Serge Lage, Iola Pinto, João Ferreira, Nuno Antunes\\
Book: Springer, Advances in Intelligent Systems and Computing volume 557\\
Date: 23 February 2017\\
DOI: 10.1007/978\textendash3\textendash319\textendash43480\textendash0\textemdash57\\
https://link.springer.com/chapter/10.1007/978\textendash3\textendash319\textendash53480\textendash0\textemdash57


% section published_paper (end)

% section publications (end)



