\abstractEN % Do NOT modify this line

MONICAP is a monitoring system for the inspection of fishing using the Global Positioning System (GPS) for vessel location and Inmarsat-C technology for satellite communications between ships and a ground control center. MONICAP was successfully introduced on the market by Xsealence and is currently installed or currently being installed on about 800 fishing vessels operating under the control of the authorities of Portugal, Spain, France, Ireland, and Angola. Within the scope of this Master's thesis, it is proposed to use Portuguese fishing data from the Vessel Monitoring Systems to extract patterns of behavior related to the fishing zones, times, speeds and directions of the course performed by the ships. The descriptive statistical analysis of these makes it possible to identify patterns of fishing activity, as well as the identification of outliers. The identification of outliers, when performed in real-time, will lead to the consequent generation of alerts. The present study represents the first comprehensive approach to detect and identify the behavior of fishing activities for the main types of equipment based on the tracking of Portuguese fishing fleet data.

% Keywords of abstract in English
\begin{keywords}
Vessel Monitoring System, Data Mining, Fishing
\end{keywords} 
