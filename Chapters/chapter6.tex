% 
%  chapter6.tex
%  ThesisISEL
%  
%  Created by Serge Lage on 2019/07/30.
%
% ================
% = Introduction =
% ================
\chapter{Conclusion}
\label{cha:conclusion}

This chapter presents the conclusion of this research, starting with an overview of the work covered in this document and concluding with a set of possible routes for future work.

\section{Overview} % (fold)
\label{sec:overview}
With the work done we were able to demonstrate several ways to classify and validate VMS data. The importance of being able to evaluate this type of data will be a great weapon against the tax fraud that occurs in the fishing sector. Another potential return from the developer's work was the possibility to understand the fishing patterns to be able to create plans for environmental protection.

\subsection{Blue Box} % (fold)
\label{sec:blue_box}

In the first solution, we were able to demonstrate that it is possible to classify in real-time two important aspects. If the vessel is fishing and if it is fishing in a new area.\\
Considering the work done we conclude that classifying if the vessel is fishing at a given moment, taking into account the historical speed of the same, it is possible since we demonstrate that the speed of each vessel has well defined the distribution of fishing speeds, thanks to the fact that the boats spend much of their time fishing.
With this information and using clustering algorithms, it is also possible to define fishing areas.\\

This type of classification is very useful to understand the fishing patterns in a given area. Another interesting result is the possibility of over the years understand if there are variations in the level of hours of fishing and fishing zones, trying to understand the temporal evolution of the fishing and by consequence of its raw material. With this to understand if the boats spend more time or less inactivity by each time they leave (it can mean that it is becoming easier or more difficult to catch fish), if there is a movement of the activity by type of license (can infer if certain types of fish are disappearing in certain areas and emerging in new areas).


% section blue_box (end)

\subsection{Server} % (fold)
\label{sub:server}
In the second solution, we want to show that it is possible to classify the fishing license by taking into account the VMS data, more precisely speed and position data.
The treatment of the data was rewarding since it was possible to find correlations between the type of fishing and the actions of the vessels in fishing activity.

It still takes a lot of work to have variables of enough quality to create a good classifying model.
I intend to create different types of data mining algorithms to determine which best fits this problem.
Finally, I intend to create an informatics system capable of receiving, classifying and verifying the VMS data of the fishing vessels.

% section server (end)

\section{Future Work} % (fold)
\label{sub:future_work}


% section future_work (end)

% chapter conclusion (end)



