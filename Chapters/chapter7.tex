% 
%  chapter7.tex
%  ThesisISEL
%  
%  Created by Serge Lage on 2019/07/30.
%
% ================
% = Conclusion =
% ================
\chapter{Conclusions}
\label{cha:conclusion}


This chapter presents the conclusion of this research, starting with an overview of the work covered in this document and concluding with a set of possible routes for future work.

\section{Overview} % (fold)
\label{sec:overview}

There is great concern about fisheries fraud as demonstrated by the Food and Agriculture Organization of the United Nations \cite{FAOFraud}.
In the European Union through Directorate-General for Maritime Affairs and Fisheries in the European commission, studies are being done and discussing the best way to act on this illegal action using new technologies like VMS \cite{WEBSITE:ECControl}.

The objective of this dissertation was to find ways to detect tax fraud in fishing activity efficiently and quickly to be able to detect when a vessel had an operation with abnormal behavior, to be able to notify the authorities while the vessel is still unloading the fish at the dock.

We were able to demonstrate two ways to classify and validate VMS data. The importance of being able to evaluate this type of data will be a great weapon against the tax fraud that occurs in the fishing sector. Another potential return from this work was the possibility to understand the fishing patterns to be able to create plans for environmental protection.

For this work, it was used real VMS data from Portuguese fisheries.

All the objectives proposed in this document have been achieved.

\subsection{Conclusions about Standalone Fishery Analysis} % (fold)
\label{sec:con_sfa}

In SFA, we were able to demonstrate that it is possible to classify in real-time two crucial aspects. If the vessel is fishing and if it is fishing in a new area.\\
Considering the work done we conclude that classifying if the vessel is fishing at a given moment, taking into account  the historic of its speed, it is possible since we demonstrate that the speed of each vessel has well defined the distribution of fishing speeds, thanks to the fact that the boats spend much of their time fishing.
With this information and using clustering algorithms, it is also possible to define fishing areas.\\

This type of classification is advantageous to understand the fishing patterns in a given area. Another impressive result is the possibility of over the years understand if there are variations in the level of hours of fishing and fishing zones, trying to understand the temporal evolution of the fishing and by consequence of its raw material. With this to understand if the boats spend more time or less inactivity by each time, they leave (it can mean that it is becoming easier or more difficult to catch fish) if there is a movement of the activity by type of license (can infer if certain types of fish are disappearing in certain areas and emerging in new areas).


The main weakness of SFA is the lack of classified VMS data. Since it is not possible to test the SFA results with data classified on the ship, it is not possible to measure the real accuracy of the classifier. With classified data, it would also be possible to adjust the classification parameters of the SFA better.

% section blue_box (end)

\subsection{Conclusions about Joined Fishery Analysis} % (fold)
\label{sub:con_jfa}
In the second solution, we want to show that it is possible to classify the fishing license by taking into account the VMS data, more precisely speed and position data.
The treatment of the data was rewarding since it was possible to find correlations between the type of fishing and the actions of the vessels in fishing activity.

It still takes much work to have variables of enough quality to create a good classification model.
We created different types of data mining algorithms to determine which best fits this problem.

The main weakness of JFA is the prospect of having fraudulent data used in training the model. This could be resolved by training the model with data on which the on-board inspection took place or giving this data more weight in the model than the un-inspected.

% section server (end)


\section{Future Work} % (fold)
\label{sub:future_work}

In future work, it is imperative to create classified VMS data to analyze the SFA accuracy better. Also, to have data from fisheries activity that was inspected by a competent authority to test the models better and if with enough data, to train them with only inspected data or giving more weight to this data so we can have more accurate models.
 

% section future_work (end)

% chapter conclusion (end)





